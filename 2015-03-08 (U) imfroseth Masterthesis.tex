\documentclass[UKenglish]{ifimaster}  %% ... or USenglish or norsk or nynorsk
\usepackage[latin1]{inputenc}         %% ... or utf8 or applemac
\usepackage[T1]{fontenc,url}
\urlstyle{sf}
\usepackage{babel,textcomp,csquotes,ifimasterforside,varioref,graphicx}
\usepackage[backend=biber,style=numeric-comp]{biblatex}

\title{Application-centric Infrastructure}        %% ... or whatever
\subtitle{Any short subtitle}         %% ... if any
\author{Ida Marie Fr�seth}                      %% ... or whoever 

\bibliography{mybib}                  %% ... or whatever

\begin{document}
\ififorside{}
\frontmatter{}
\maketitle{}

\chapter*{Abstract}                   %% ... or Sammendrag or Samandrag
ThIS Is a change


The text of the abstract;
typically, 2--5 sentences.

What it is all about??
Title page, abstract, ...

1. Introduction, containing: short intro into the area, what is happening
1.1 Motivation, containing: what triggered me to write about what I'm writing about
1.2 Methods, containing: which methods are you using, how do you apply them
2. Scenario, optional chapter for explaining some use cases
one main scenario, and applicability of others; what is the outcome of the scenario ("switching action", display information, ...)
2.1 user scenario, (bad name, needs something bedre)
2.2 Requirements/Technological challenges, here: Semantic technologies, User Profiles/preferences, context-aware services, ... WLAN provisioning (FON.com), reasoning towards a group profile (advertisement, entertainment, information)
3. State-of-the art/Analysis of technology, structure your content after hardware/SW (or other domains). Describe which technologies might be used to answer the challenges, and how they can answer the challenges. Here: Middleware for bindling the moduls together. Focus on engine, Web or mobile?
for all subchapters: provide an overview, then evaluate ("what is important for me?"), recommendation on which solution to go for
3.1 technology A: Semantic technologies, Protege, Ontology
3.2 technology B: User profiles (survey: SPICE, FOAF, tourist ontology,... -> none of them fit to my needs. Extend FOAF in areas of .....)
3.3 Context awareness:
3.4 Reasoning: overview, "why SWRL?"
4. Implementation
4.1 Architecture, functionality
4.2
5. Evaluation
5.1 Security (man in the middle attacks, replay attacks...)
5.2 Group Profile based service Personalization
5.3 Future work
6. Conclusions
References

\tableofcontents{}
\listoffigures{}
\listoftables{}

\chapter{Preface}                    %% ... or Forord
When and Where
This thesis was written as a part of my Master?s degree in Informatics at the University of Oslo. Most of the work was done in the period from Januar 2016 to May 2017. The thesis was done in collaboration with Norwegian Defence Research Establishment, but the thesis have solely been written by the author. 

	- in coporation with Norwegian Defence Research Esablishment
How literature have been examined and where the publications have been found
The names of the supervisors
	- Mariann Hauge
	- Erlend Larsen
	- ?
Acknowlegement
	- Jan Erik Voldhaug
	- Petter Kristiansen
	- Ingar Bentstuen
	- review:  Gunstein Thomas Fr�seth(?)
	- Lasse S�ther

The envisaged time schedule (for a long thesis/60 ECTS) is:
T0 0 starting month, T0+m denotes the month where the contribution to a certain chapter shalle be finalized
T0+2 months: create an initial page describing the scenario
T0+3: Provide a list of technologies which you think are necessary for the thesis
T0+4: Establish the table of content (TOC) of the envisaged thesis. Each section shall contain 3-10 keywords describing the content of that section
T0+7: Provide a draft of section 2 (scenario) and 3 (technologies)
T0+10: Establish a draft on what to implement/architecture
T0+11: Set-up an implementation, testing and evaluation plan
T0+15: Evaluate your solution based on a set of parameters, keep in mind there is no such thing as a free lunch
T0+17: Deliver the thesis

\mainmatter{}
\part{Introduction}                   %% ... or Innledning or Innleiing

\chapter{Background}                  %% ... or 
%\subchapter{Software defined Networks}
\input{SdnTheory}
\part{The project}                    %% ... or ??

\chapter{Planning the project}        %% ... or ??


\part{Conclusion}                     %% ... or Konklusjon

\chapter{Results}                     %% ... or ??


\backmatter{}
\printbibliography
\end{document}
