%!TEX root = ../xxxx-xx-xx_idamfro_masterthesis.tex
\chapter{Introduction}
This thesis investigates how \gls{cots} products, in the form of handheld smart devices, can be used as a cheap and powerful tool for situational awareness in military operations. Traditionally, military information systems are proprietary and expensive due to various special requirements, but there has been a shift towards using \gls{cots} the last couple of years due to the concept of \gls{nnec}.

The main focus in this thesis has been on usability, and how to best represent rich and complex content in a simple and intuitive way. The thesis also addresses other related issues like power consumption and network considerations through a requirements analysis. The findings are implemented in a working prototype which is evaluated through a \gls{he} and a real world user test. The results shows that the implemented solution .....

**MA KANSKJE INKLUDERE AT ARBEIDET BYGGER PA SMART...

%\section{Background}

%\section{Motivation}
%* Situational awareness = key enabler for military suveregnity

%* Challenges with mobile touch screen development. 

\section{Background and motivation}
\subsection{Situational awareness in military operations}

\subsection{Related work}
\subsubsection{The Tactical Ground Reporting System (TIGR)}
(USA) 2005-2007
* Cloud based
* Web client on different platforms
* Distributed servers which seems like one
\cite{04-Evans-tigr}\cite{04-Ewy-tigr_iraq}

\subsubsection{Nett Warrior} 
NETT WARRIOR -> Android but all communication functionality in the device is disabled, only connected throgh USB to a military radio

Own version of Android which is security 

\subsubsection{}

In the concept of \gls{nnec}, or the Norwegian equivalent \gls{nbf}, military sovereignty is achieved through network centric collaboration.  The concept emphasizes information sharing, doctrinces, policy and procedures as key enablers for effective command and control. 

information and technology. 

The correct information to the correct time. 

Military operations are usually very dynamic, and to be able to effectively share a common operational picture. --- have the current situational picture is often crucial for leaders to make good decisions. 

the main task is not to know the current situation details, but act on them.  


In this concept 


 with people, doctrines and procedures as the main drivers, and technology as the key enabler%\cite{NBF}\cite{NNEC}.

NNEC - NBF => The correct information to the right time.

Situational awareness is a key enabler for military sovereignty. 

Military communication and information systems have special requirements regarding functionality, security and robustness which have lead to expensive proprietary solutions. At the same time, there is a need for powerful communication and sensor systems even at the lowest level to achieve good situational awareness. With the emerge of smart technology and the Long-Term Evolution standard, the consumer can get multi-sensor devices at low cost. These technologies have, over the last couple of years, gain a growing interest within the military community.


((ALTERNATIV1))
The SMART project by the \gls{ffi} investigated how mobile smart devices can be used as a simple and cheap tool for situational awareness to mobilization units in military operations. This thesis is a continuation of this work and involves analyzing the requirements for such apps, investigating frameworks, tools and techniques to meet these requirements, and evaluating the implementation through various tests. 

((ALTERNATIV2))
This thesis investigates how mobile smart devices can be used as a simple and cheap tool for situational awareness to mobilization units in military operations. Since the mobilization units have very limited time for training, the primary focus has been on the design of an easy to use and content rich user interface. Another concern when developing smartphone apps, and especially for use in military operations, is battery consumption, and the thesis investigates some techniques to optimize the battery lifetime. 

((DETALJENE ANG HVA DENNE OPPGAVEN DEKKER KOMMER KANSKJE TIL AA BLI ENDRET LITT ETTERVHER))


* NBF = informasjonsbehov

* Mobilization units

* 

%Mobile smart technology have had a rapid growth the last two decades, and the consumer can get relatively powerful multi-sensor devices at low cost. With open and well documented APIs and the App store distribution model, every person with basic programming skills can develop and distribute rather complex and specialized applications. The last years even apps for situational awareness in  military operations have started to enter the market \cite{dina}. 

Military \gls{c2is}s have special requirements regarding functionality, security and robustness, and many operations are executed in areas with no communication infrastructure or in areas where you can not trust the provider of the infrastructure. These issues are reasons why smart technology is not fully adopted in military operations yet, however there is a growing number of experiments and even commercial products\cite{dina} which try to customize smart technology to fit these requirements. 

The research in the area is diverse from trying to solve connectivity issues with \gls{lte} or WiFi direct \footnote{During the \gls{nato} exercise Bold Quest 2013, multiple nations deployed their own \gls{lte} network see https://www.dvidshub.net/news/113548/military-tests-4g-lte-technology-during-bold-quest-132} \cite{04-Do_et_al-Ad_hoc_nw_HybCAST}\cite{04-Kaul_et_al-adaption_of_smart_network}, to adding security. Apart from publicly available work, there are also several restricted project whose details cannot be included in this thesis. Furthermore, the commercial available situational awareness apps are built for the professional soldier and uses jargon and includes features which require training\cite{ep1667_bakgrunn}. This makes these apps less suited for typically mobilization units which may train only a week a year, which is what the SMART project at the \gls{ffi} tries to solve. 

The SMART project by the \gls{ffi} is an \gls{cde} activity which examines how mobile smart technology can be used as cheap but powerful devices to provide situational awareness for mobilization units with low resources and training\cite{ep1667_bakgrunn}. Through the project a demonstrator was developed consisting of a security component, a sever back-end system and an smartphone app based on the Android platform. The system was tested through real world user tests, and the results shows that the system has great potential for such application. Due to the limited time of the activity, little focus was put on the optimization and the design of the system. This thesis will analyze the current solution of the SMART demonstrator and the domain which will form a set of prioritized requirement for such an app. A selected  will give a set of requirement, propose a set of changes to the SMART prototype to meet these requirements.  Some selected requirements will be implemented, tested and the impact will be measured to propose some basic guidelines when developing smart apps for military operations.

\section{Problem statement}
***NOTES***
\begin{itemize}
\item{This thesis will continue the work in the SMART project and investigate further how the Android platform can be optimized to fit in military application.}
\item{Requirements}
\item{Design guidelines}
\item{Optimizing}
\end{itemize}

\subsection{Goals}
Prosessmål

Resultatmål
\begin{itemize}
	\item{Develop a working prototype}
	\item{}
	\item{}
\end{itemize}

\section{Premises}

\section{Scope and Limitations}

\section{Research Methodology}
Computer science can be described as an interdisciplinary field with three different paradigms \cite{01-Denning-computer_science,01-Denning-computing_as_a_dicipline}:
%\cite{01-Denning-computing_as_a_dicipline} and \cite{01-Denning-computer_science} describes computer science as an interdisciplinary field with three different paradigms: 
\textit{Theory, Experimentation} and \textit{Design}. While the first two paradigms are concern with building broad conceptual frameworks and models within the entire field or a domain \cite{01-Denning-computer_science}, the third paradigm is about \textit{Constructing computer systems that support work in given organizations or application domains} \cite{01-Denning-computer_science}. The work within this thesis fit within the last paradigm.

% Furthermore, Denning \cite{01-Denning-computer_science} divides the field into 12 subfields, and the relevant subfields for this thesis are Architecture, Software Engineering, \gls{hci} and Organizational Informatics. 
Design is usually performed as an iterative process with the four steps \cite{01-Denning-computing_as_a_dicipline} listed in bold typeface in the listing below. The work performed in this thesis that maps to each of the steps are found in regular typeface to the right of the listed step. Within the scope of this thesis only one iteration was performed, however, since the work continues the SMART project it can be viewed as a second iteration. 

\begin{description}
	\item[State requirements:] During the first phase the business case and current solution was analyzed. This work ended in a prioritized list of requirements found in chapter \ref{sec:analysis_summary}.

	\item[State specification:] Based on the priority of the requirements found, a subset was chosen and a study of related work and state-of-the art were investigated to form a foundation for the specification of a new design and is covered in chapter \ref{sec:analysis_summary} to \ref{chap:design}.

	\item[Design and implement:] Based on the findings in phase one a new design were proposed and implemented. This work is described in chapter \ref{chap:design}.
	
	\item[Test the system:] Qualitative methods have been used to evaluate the system. First an \gls{he} was performed using mobile specific heuristics and secondly a user test with questionnaires and interviews. The user test was performed as close as possible to a real world situation, by having 50 soldiers from the \gls{hv} using the system as a tool during a regular military exercise.
%\item{\textbf{Design and prototyping:} }
%\item{\textbf{Testing and data collection:} To measure how successful the implementation have been a set of tests and experiments were conducte{}d. The tests included both technical lab tests and practical field test based on the type of requirement which were implemented.}
%\item{\textbf{Evaluation:} the gathered data was analyzed and evaluated and a set of guidelines on what to consider when developing apps for situational awareness for military operations were proposed, in addition to future work.
%**INCLUDE SOME FINDINGS??**

\end{description}


\section{Contribution}

\section{Thesis outline}
The rest of this thesis is organized as follows:
In chapter \ref{chap:analysis} the reader will get familiar with the background for the work, including related work, an requirement analysis and relevant technologies and frameworks. Then the design of the developed prototype is described in chapter \ref{chap:design} using a top-down approach. In chapter \ref{chap:evaluation} the prototype is evaluated based on a user test and an \gls{he}. Chapter \ref{chap:conclusion} concludes the work and propose some future work to be performed.


